\documentclass[a4paper]{jarticle}
\usepackage[dvipdfmx]{graphicx}
\usepackage[top=30truemm,bottom=30truemm,left=25truemm,right=25truemm]{geometry}
\usepackage{ascmac, here, txfonts, txfonts}
\usepackage{listings,jlisting}
\usepackage[dvipdfmx]{color}

\lstset{
	%プログラム言語(複数の言語に対応,C,C++も可)
 	language = C++,
 	%背景色と透過度
 	backgroundcolor={\color[gray]{.90}},
 	%枠外に行った時の自動改行
 	breaklines = true,
 	%自動改行後のインデント量(デフォルトでは20[pt])
 	breakindent = 10pt,
 	%標準の書体
 	basicstyle = \ttfamily\scriptsize,
 	%コメントの書体
 	commentstyle = {\itshape \color[cmyk]{1,0.4,1,0}},
 	%関数名等の色の設定
 	classoffset = 0,
 	%キーワード(int, ifなど)の書体
 	keywordstyle = {\bfseries \color[cmyk]{0,1,0,0}},
 	%表示する文字の書体
 	stringstyle = {\ttfamily \color[rgb]{0,0,1}},
 	%枠 "t"は上に線を記載, "T"は上に二重線を記載
	%他オプション:leftline,topline,bottomline,lines,single,shadowbox
 	frame = TBrl,
 	%frameまでの間隔(行番号とプログラムの間)
 	framesep = 5pt,
 	%行番号の位置
 	numbers = left,
	%行番号の間隔
 	stepnumber = 1,
	%行番号の書体
 	numberstyle = \tiny,
	%タブの大きさ
 	tabsize = 4,
 	%キャプションの場所("tb"ならば上下両方に記載)
 	captionpos = t
}

\setlength{\topmargin}{-0.3in}
\setlength{\oddsidemargin}{0pt}
\setlength{\evensidemargin}{0pt}
\setlength{\textheight}{46\baselineskip}
\setlength{\textwidth}{47zw}

\makeatletter

\def\@thesis{個人開発用仕様書}
\def\id#1{\def\@id{#1}}
\def\department#1{\def\@department{#1}}

\def\@maketitle{
\begin{center}
{\huge \@thesis \par} %修士論文と記載される部分
\vspace{10mm}
{\LARGE\bf \@title \par}% 論文のタイトル部分
\vspace{10mm}
{\Large \@date\par}	% 提出年月日部分
\vspace{20mm}
{\Large \@department \par}	% 所属部分
{\Large No. \@id \par}	% 学籍番号部分
\vspace{10mm}
{\large \@author}% 氏名
\end{center}
\par\vskip 1.5em
}

\makeatother

\title{Pokemon Card Game}
\date{2018. . }
\department{}
\id{}
\author{}

\begin{document}
\maketitle
\newpage
%目次の表示
\tableofcontents

%表目時の表示
\listoftables

% 図目次の表示
\listoffigures
\newpage
%章
\section{名称}
\begin{table}[H]
  \begin{tabular}{cc}
    ゲーム名称 & 「Pokemon Card Game -PKC-(ポケカ)」\\
    &\\
    機能1 & vs AI \\
    &\\
    機能2 & vs Friend \\
    &\\
    機能3 & vs Deck manager\\
    &\\
    機能4 & vs Card List \\

  \end{tabular}
\end{table}

%章
\section{参考イメージ}
  Pokemon Trading Card Game Onlineの日本語版及びUI等はオリジナルにしたい。

\section{概要}
  ポケモントレーディングカードゲーム(以下、ポケカ)をソフトウェアとして実装し、ポケカユーザーの支援を行う。
  基本的な機能は以下のとおりである。
  \begin{itemize}
    \item AIとの対戦
    \item 対人戦(オフライン)
    \item デッキ構築
    \item カード図鑑
  \end{itemize}

  機能1は一人でも対戦を楽しむことができ、また新戦術の試運転にも活用できる。
  機能2はあたかも友人と実際のカードでポケカ対戦を行っているかのような気分になれる。
  機能3は対戦に不可欠なデッキの作成を行える。
  機能4はコレクターにはたまらない歴代カードの図鑑である。

%参考文献
\begin{thebibliography}{n}
	\bibitem[opt1]{key1} 文献情報
	\bibitem[opt2]{key2} 文献情報
\end{thebibliography}

%改頁
\newpage

%付録
%\section*{付録}
%	測定データの一次資料等を付録として添付する。
%
%	%画像の張りつけ
%        %画像はターミナルにて、$ ebb file_name して.bbファイルを作成する
%	\begin{figure}[htbp]
%		\begin{center}
%			\includegraphics[width=0.9\textwidth]{file path}
%			\caption{図の名前}
%			\label{fig:winter}
%		\end{center}
%	\end{figure}
%%改頁
%	\newpage

\end{document}
